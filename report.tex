\documentclass{article}

\usepackage{url}
\usepackage{graphicx}
\usepackage{listings}
\usepackage{verbatim}
\usepackage{xspace}

\usepackage{float}
\usepackage{hyperref}
\usepackage[english]{babel}
\usepackage[numbers]{natbib}
\usepackage{amsmath}


\setlength{\textfloatsep}{5pt plus 1.0pt minus 2.0pt}
\setlength{\floatsep}{7pt plus 1.0pt minus 2.0pt}
\setlength{\intextsep }{7pt plus 1.0pt minus 2.0pt}

\setcounter{secnumdepth}{4}
\setcounter{tocdepth}{4}


\begin{document}
\renewcommand{\bibname}{References}

\pagestyle{plain}
\pagenumbering{roman}
\setlength{\tabcolsep}{10pt}

\title{Linked Open Data at University of Technology Vienna}

\author{Stefan Gamerith\\
\texttt{e0925081@student.tuwien.ac.at} \\ Student ID.: 0925081}

\maketitle

\begin{abstract}
	%Here comes the abstract%
\end{abstract}

\tableofcontents

\newpage
\pagenumbering{arabic}

\section{Introduction}
While the pressure on governments and public organizations to release \textit{open data} has significantly grown with the spread of information systems, there has been also a need for \textit{linking} these data from various sources to understand the information as a whole.

Open Data includes non-privacy-restricted and non-confidential data. Therefore any restrictions in distribution are prohibited and data is funded only by public money.~\citet{article:janssen2012benefits} The application domain for Open Data providers is not restricted by its nature in any way and ranges from traffic, weather, statictics to budgeting in the public sector. Just the publication of Open Data seems not enough but in addition the implementation of a feedback loop result in \textit{Open Government}. This has the advantage of a constant adaption to the citizen's needs instead of just visualizing former closed data. 

Despite its wide adoption of Open Data it does restrict the published Data format in any way, thus complicating the integration of heterogeneous data sets. The World Wide Web has proven great success spreading knowledge of various data sources all over the world. The building block of the Web are documents and links connecting them to form a global information space. This can be seen as the key success factor in its nearly unconstrained growth~\cite{report:jacobs-i-2004--a}. 
Following these principles of publishing and connecting data on the Web is known as \textit{Linked~Data}. More technically, it refers to machine-readable data which is linked to external data sets and can in turn be linked from other data sets. 

\citet{artivle:bernerslee-t-2006-1} developed a five star rating scheme for classifying \textit{Linked Open Data} which combines Linked Data and Open Data. The scheme ranges from one star describing Open Data only to five stars describing Open Data in a machine-readable format using open standards with links to other data sets. 

%% The focus of this paper, though, is the applicability of Linked Open Data at the University of Technology in Vienna. 


\newpage
\bibliographystyle{plainnat}
\bibliography{references}

\end{document}
