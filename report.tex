\documentclass{article}

\usepackage{url}
\usepackage{graphicx}
\usepackage{listings}
\usepackage{verbatim}
\usepackage{xspace}

\usepackage{float}
\usepackage{hyperref}
\usepackage[english]{babel}
\usepackage[numbers]{natbib}
\usepackage{amsmath}
\usepackage{csquotes}
\usepackage[final]{pdfpages}
\usepackage[inline]{enumitem}

\setlength{\textfloatsep}{5pt plus 1.0pt minus 2.0pt}
\setlength{\floatsep}{7pt plus 1.0pt minus 2.0pt}
\setlength{\intextsep }{7pt plus 1.0pt minus 2.0pt}

\setcounter{secnumdepth}{3}
\setcounter{tocdepth}{3}


\begin{document}
\renewcommand{\bibname}{References}

\pagestyle{plain}
\pagenumbering{roman}
\setlength{\tabcolsep}{10pt}

\title{Linked~Open~Data at University of Technology Vienna}

\author{Stefan Gamerith\\
\texttt{e0925081@student.tuwien.ac.at} \\ Student ID.: 0925081}

\maketitle

\begin{abstract}
	%Here comes the abstract%
\end{abstract}

\tableofcontents

\newpage
\pagenumbering{arabic}

\section{Introduction}
While the pressure on governments and public organizations to release \textit{Open~Data~(OD)} has significantly grown with the spread of information systems, there has been also a need for \textit{linking} these data from various sources to understand the information as a whole.

Open Data includes non-privacy-restricted and non-confidential data. Therefore any restrictions in distribution are prohibited and data is funded only by public money.~\citet{article:janssen2012benefits} The application domain for Open Data providers is not restricted by its nature in any way and ranges from traffic, weather, statictics to budgeting in the public sector. Just the publication of Open Data seems not enough but in addition the implementation of a feedback loop result in \textit{Open Government}. This has the advantage of a constant adaption to the citizen's needs instead of just visualizing former closed data. 

Despite its wide adoption of Open Data it does restrict the published Data format in any way, thus complicating the integration of heterogeneous data sets. The World Wide Web has proven great success spreading knowledge of various data sources all over the world. The building block of the Web are documents and links connecting them to form a global information space. This can be seen as the key success factor in its nearly unconstrained growth~\cite{report:jacobs-i-2004--a}. 
Following these principles of publishing and connecting data on the Web is known as \textit{Linked~Data~(LD)}. More technically, it refers to machine-readable data which is linked to external data sets and can in turn be linked from other data sets. 

\citet{artivle:bernerslee-t-2006-1} developed a five star rating scheme for classifying \textit{Linked Open Data~(LOD)}, which combines Linked Data and Open Data. The scheme ranges from one star describing Open Data only to five stars describing Open Data in a machine-readable format using open standards with links to other data sets.

Although Linked~Open~Data offer universities new opportunities for providing unprecedented insight into its core activities and ease application development, a major \textbf{problem} is that \textbf{Linked~Open~Data has not been widely adopted by universities yet}. Even tough there are a few examples~\cite{url:linked-universities-members} of publishing university related data as Linked~Open~Data, there has been little knowledge of using Linked~Open~Data for publishing university related information. 

The remainder of this section states the addressed research question of this paper, describes the contributions in the course of investigating the research questions and gives an overview of the structure of this paper.

\subsection{Research Questions}
The fundamental research questions underlying this paper is:
\begin{displayquote}
\textit{How can Linked~Open~Data help to improve processes in university context and how can it be successfully applied?}
\end{displayquote}
More concrete this paper concentrates on the following more concrete research questions:
\paragraph{Q1: What are best practices regarding the applicability of Linked Open Data in university settings?}
As of now, there are no established best practices for the use of Linked Open Data due to its little adoption in university contexts. For this very reason it is crucial to identify strengths and limitation from previous experiences~\cite{url:linked-universities-members} of using Linked~Open~Data as core technology. 
\paragraph{Q2: What are major benefits and barriers for each stakeholder and what are useful use cases?}
We identified three different stakeholders in university context \textit{Students}, \textit{Researchers} and \textit{Administration staff}. Since the success of any new technology highly depends on the acceptance of the stakeholders, the needs of each of the target groups needs to be examined. Furthermore, use cases are important to showcase profits and shortcomings to a non-technical audience. 
\paragraph{Q3: What are major challenges for the implementation of a Linked Open Data~solution?}
As the implementation of a Linked Open Data solution is a time consuming task, the knowledge of probable challenges from the technical perspective as well as from the management perspective is a key factor for the successful adoption. 
\paragraph{Q4: How would a prototypical implementation of Linked~Open~Data look like?}
Among the various existing data sets available it needs to be investigated if a (semi-) automatic transformation is feasible or is the manual data provision enough. In addition, from an implementers perspective of view, critical factors regarding the storage and retrieval of Linked~Open~Data need to be identified. 

\subsection{Methodology}
Finding an answer for the research questions above has lead to the following three methodologies:
\paragraph{A coordinated set of semi-structured interviews}
To answer research questions~(RQ) two to four, we interviewed a selected set of stakeholders representing \textit{Students}, \textit{Researchers} and \textit{Administration staff} respectively. Semi-structured interviews were selected as the means of data collection because they are well suited for exploring the impressions and interests of the interviewees as in a discussion while still following a defined structure. 
\paragraph{Litrature Review}
Undertaking a litrature review to justify scientific contributions and making sound conclusions is an established practice in any scientific community. Since our scientific work targeted in particular to the Semantic Web community, we made some pre-assumptions of a basic understanding of the technologies and concepts regarding Linked Data. More specifically, the concept of an ontology and example knowledge descriptions languages describing these will not be covered in this paper. 
\paragraph{Conceptual System Design}
The development of applications based on Linked Open Data requires a methodology which describes a common understanding of the overall system infrastructure. Therefore we designed a conceptual model of a prototypical implementation of a Linked Open Data solution. 

\subsection{Contributions}
The work in this paper mainly contributes to different aspects wich need to be considered when designing and implementing a Linked Open Data application.
More precisely, our contributions can be categorized into the following four areas:
\paragraph{1. Identifying best practices for Linked Open Data in university context.}
Due to the crowing complexity and the large amount of data information systems need to process, there has been the need to efficiently handle Linked Data as well. We gave a brief overview of the already published research work regarding Linked Open Data in university context. In particular, we compared the profits and shortcomings in existing Linked Open Data solutions. 
\paragraph{2. Finding benefits/barriers with additional use cases for stakeholders.}
As with every software project the very first phase of the Software Development Lifecycle~(SDLC) is the \textit{Evaluation of the Requirements}. As a Linked Open Data solution has additional requirements to the structure of the data and due to its open nature, we investigated if the overhead compared to an established technology (e.g. a database based solution) is worth the effort. A set of selected participants from the areas Research, Student~Affairs and Administration are interviewed at the University of Technology in Vienna and their benefits/barriers are compared. Additionally, we proposed several use cases emphasising their point of view. 
\paragraph{3. Discovering possible obstacles for implementers of a Linked Open Data Solution.}
As the application domain for a Linked Data is limited to the university context, our work includes a defined set of Linked Open Data applications which were merged together from the conducted interviews. That use cases showcased probable shortcomings which might arise before, during or after the implementation.
\paragraph{4. Sketching a prototypical implementation of a Linked Open Data Solution.}
In consideration of the above mentioned obstacles of a possible Linked Open Data Solution, we gave an outline of a prototypical implementation. It begins by covering the whole process of data provision and ends by applications made for end users. 
\subsection{Structure of this Paper}
\%\%\%\%tbd\%\%\%\%

\section{Related Work}
\%\%\%\%tbd\%\%\%\%
%\newpage
\section{Benefits and Challenges}
In this chapter we figure out the driving factors of publishing university related assets as Linked Open Data and introduce possible example applications built on top of it. First, we describe the used methodology to show benefits and challenges of using Linked Open Data at the University of Technology in Vienna. Second, we discuss the concrete topics of the interview, hence, potential needs, benefits and challenges are examined. Last, we compare profits against shortcomings by giving concrete use-cases to show under which circumstances Linked Open Data solutions are preferable to traditional ones (e.g. a database based implementation). 
\subsection{Methodology}
In this sub-section we describe the examined university and the interviewed people more thoroughly, elaborate on the interview design, analysis and threats to validity.
\subsubsection{The interview context}
All interviews were taken at the University~of~Technology in Vienna~\cite{url:university-of-technology-vienna}. The university was founded in 1815 as Imperial and Royal Polytechnical Institute. 160 years later~(1975) the university was renamed to its current name \textit{University of Technology}. At the time of writing the majority of the faculties are located at the fourth district in Vienna with nearly 9,000 rooms spreading over 276,000 square meters.

The university employees can be divided into the following three categories: \textit{administrative staff}, \textit{students} and \textit{researchers}. Since a detailed investigation of all three areas would exceed the scope of this work, the area of interest for this paper are administrative people. There exists a dedicated paper for each of the other two categories~(\%\%\%LINK TO THE OTHER PAPERS\%\%\%), though. 

The university provides a website~\cite{url:university-list-of-org-units} of all organizational units falling into the category Administration~\&~Services. For the evaluation of this work several departments were contacted by email. However, the majority of them declined participating because either they did not feel comfortable answering questions because of the technical concentration of the interview (e.g. Semantic Web in general and Linked Data in particular) or they were too busy in the restricted time frame~(4~months) were the interviews were planned. However, we were forwarded to representatives of the organizational units \textit{Research and Transfer Support}~\cite{url:university-research-transfer} and \textit{Department for Studies and Examinations}~\cite{url:university-studies-and-examinations}. 
\subsubsection{Interview design}
As a solid preparation phase is the foundation of a hight quality interview-outcome, a considerable amount of time should be invested. Conceptually there exist three different types of interviews: \textit{Unstructured interviewing}, \textit{Semi-structured interviewing} and \textit{Fully-structured interviewing}~\cite{book:bernard-antropology-semi-structured-interview}. 
Due to the nature of unstructured interviews, they are comparable to discussions, not following specific guidelines. In contrast, data collection in semi-structured and fully-structured interviews is typically documented by questionnaires. \citet{article:harris2010mixing} described the difference between these two interview types as, while in fully-structured interviews the participant responds to predefined answers~(e.g. choosing from options), semi-structured interviews also focusing more on testing interviewees responses, motivating them to provide more background information and clarification. The interviews conducted for this paper were of a semi-structured nature because of the small number of participants, centered on understanding the needs of the interviewees~\cite{book:miles2005handbook}. 

The questionnaire is composed of three major parts:
\begin{enumerate}
	\item General questions targeted at getting a better understanding of the participants. It includes exploring prior knowledge about Linked Open Data in particular and information systems in general. 
	\item Analysing their thoughts and opinions about presented example applications built using Linked Open Data, motivating them to provide use cases for new applications. 
	\item Exploring existing and additional sources for data provision. 
\end{enumerate}


\subsubsection{Data Analysis and Validity}
Semi-structured interviews mix open and closed questions, allowing qualitative and quantitative analysing methods for examining the outcome of the interview~\cite{article:runeson2009-interview-guidelines}. Quantitative data analysis is measurable and objective, typically generating numerical values and charts. Qualitative data analysis is descriptive and subjective, focusing on derivating the results and driving conclusions from the collected data~\cite{article:yin2003case-case-study-research-and-design}. Due to the exploratory nature of the interviews a detailed description of qualitative and quantitative analysis methods were omitted here.  

One common threat to the validity of the collected data is a biased view on the answers given during an interview. Two countermeasures were taken to reduce this risk. Frist, the interviews were conducted by multiple researchers. Second, the interview was recorded for later analysis. 

\subsection{Results}
After giving a detailed description of the used methodology in the previous sub-section, this sub-section contains the outcome of the conducted interviews. First, potential use cases for building applications and publishing Data as Linked Open Data are identified including possible benefits and barriers for the stakeholders. Second, benefits and challenges are compared with one another resulting in identified guidelines for the use of Linked Open Data. 
%%%%%%%%%%%%%%%%%%%%%%%%%%%%%%% USE CASES %%%%%%%%%%%%%%%%%%%%%%%%%%%%%%%%%
%Listing of potential use cases (Applications/Data)%
	%Inline advantages/disadvantages for each use case

\subsubsection{Building Map}
At the time of writing this paper there exist a website\footnote{\url{http://www.wegweiser.ac.at/}} for viewing lecture rooms, offices and other geographical related data. These website includes content of all austrian universities, hiding detailed information as elevators, stairways and other points of interest as buildings near-by, pubs, coffee~shops. 
This data is particularly relevant for external visitors who are new to the campus. Furthermore, maps are published as a PDF-file or image, making it impossible to include any real~time data such as the utilization of computer rooms and defect elevators.

To give interviewees a better idea of a building map figure~\ref{fig:um-map-app} shows a map application built for the University of Economy in Vienna. 
\begin{figure}[H]
	\centering \includegraphics*[width=.8\columnwidth]{map_wu_wien.png}
	\caption{Map application of the University of Economy in Vienna}
	\label{fig:um-map-app}
\end{figure}
Although there there is no indication if the map uses Linked Data as its source, the user interface shows everything of what is achievable with Linked Open Data. 
The user interface shows a list of selectable points of interest with searching capabilities. In this concrete example a search for the rector of the University of Economy in Vienna has been initiated, highlighting the rectors office and selecting the appropriate floor on the left side. In addition, a guided route is initiated by clicking an adequate link on the turquoise overlay box. 
\paragraph{Benefits}
All interviewed stakeholders responded positive to the idea of a building application described as above. In particular, they liked the ability of highlighting specific rooms and navigating through different floors of a building. In general, there are advantages for disabled people and external visitors as outlined in the beginning of this sub-section. 

\paragraph{Challenges}
Digital data is the foundation of information systems, requiring approaches for converting non-machine-readable data to machine-readable formats. Unfortunately the Department of Construction and Technology\footnote{\url{http://www.gut.tuwien.ac.at/}} at the University of Technology in Vienna hosts only hosts buildings plans build by Computer aided Software, not adding required metadata for further processing. 

Due to the openness of Linked Open Data special attention needs to be taken to data ownership and copyright issues. Despite the fact that the austrian government passed a law\footnote{Informationsweiterverwendungsgesetz (IWG)} regulating how and which governmental data and documents could be published, there are still legal factors which need to be clarified, notably if data ownership is exclusively hold by public institutions. 
\subsubsection{Tool for statistical evaluation of student related data}
Unistats\footnote{\url{https://unistats.direct.gov.uk/}} has made statistical course data from universities and colleges across United Kingdom available. These data is published under the Open Government License\footnote{\url{http://www.nationalarchives.gov.uk/doc/open-government-licence/version/2/}} allowing copying, publishing, distributing and transmitting the data without restrictions. Besides offering data about undergraduate courses, post-graduate courses, students satisfaction scores, salaries after study and other similar information, a course assistant supports prospective students in choosing the appropriate study. 

Figure~\ref{fig:unistats-uk} shows suggestions of the course assistant resulting from an example search query. The example query selects all part time, computer science courses in England  ending in a Bachelor degree with a sandwich year. 
\begin{figure}[H]
	\centering \includegraphics*[width=.8\columnwidth]{unistats_uk.png}
	\caption{Course assistant application of Unistats UK}
	\label{fig:unistats-uk}
\end{figure}

\paragraph{Benefits}
All interviewees found the course assistant useful. The application is particularly attractive future students who are uncertain about which course to take. On the other side participants had concerns about major challenges listed below.  
\paragraph{Challenges}
As one dominant data source for the course assistant are evaluation data from students feedback, the total number of received feedback is shown. Unfortunately there is often not enough feedback data available to draw conclusions from students feedback. This seems a rather general problem of feedback related data analysis. 

Another point of criticisms are regulations of publishing students feedback. Despite the fact that the University of Technology in Vienna do offer mechanisms for rating courses, at the time of writing this paper feedback related data is not accessible from outside of the universities backbone net. 
\subsubsection{Researcher Portfolio}
The University of Technology in Vienna provides researchers a platform\footnote{TISS Forschungsprofil \url{https://tiss.tuwien.ac.at/fpl/}} for presenting their research portfolio. The intention is give external stakeholders an insight of the research profile of individual researchers, institutes, organizations and university wide. However, searching for specific topics or cooperations from a non-technical perspective is at the time of writing this paper not possible. 

\citet{article:publication-database-linked-data} provides showcases of successful application of publication data sets realized with Linked Data. Even tough a publication database built using Linked Data has advantages over the existing one, there are some challenges wich need to be taken into account when deciding for these technology. 
\paragraph{Benefits}
Scientific research often involves cooperation with third parties limited at a specific research domain. A Linked Data solution easily brings researchers and partners together as similar competencies merged under topics understandable for non-technicians. 

Linked Data relies on open standards, simplifying building of third party applications (e.g. charts). Participants responded that expressive data visualization helps in identifying cooperations between research and external stakeholders.  
\paragraph{Challenges}
As with any database keeping data up-to-date is almost in any case challenging. In particular if there is no way of (semi-)~automatic data transformation. Even though a central database would remove the burden of duplicate data sets on different places, the problem of manual data provision still remains. 

Legal regulations of privacy aspects is different from country to country, making the maintenance of a global publication database difficult. Privacy regulations often prohibits data distribution even in restricted areas. 
\subsubsection{Summary}
In this section we described our methodology of identifying benefits and challenges in designing applications using Linked Open Data. We used a technique called semi-structured interviewing for data collection. Furthermore, we gave arguments for data validity and discussed different data analysation methods. Since our aim is to better understand needs and challenges for stakeholders, the majority of questions are of open nature, qualifying for qualitative data analysis. 

In the remainder of this section we gave use cases for Linked Data applications and identified benefits and challenges in design and implementation. In addition to prepared use cases we listed additional use cases suggested by interviewees. Among the individual benefits for each use case we found challenges common to all proposed applications. Privacy and legal issues are major challenges not only to Linked Data but information systems in general. This is particularly true for Open Data. Thus, thoroughly clarifying data ownership and possible legal obstacles are key factors in publishing Open Data. 

\section{Proposed technical Architecture}
As an organization covering areas like research, student affairs and administration a university has to manage a significant amount of knowledge, adding new information on a daily basis.
Such an application domain is complex and includes areas like management of an academic library and provision of educational resources which have to be conform with stakeholders requirements. Traditionally the \textit{Service Oriented Architectures~(SOA)} have been used to meet these needs. However, as the application domain grows many small and similar services tend to emerge. That phenomena can not only be observed at the Vienna University of Technology, but also at the Open University\footnote{\url{http://www.open.ac.uk/}}~\cite{inproceedings:zablith_consuming_2011}.

A major problem of evolving similar, independent services are diverging data formats and service owners. Thus, knowledge and administrative information that has been collected by multiple services can not be easily interlinked. An example for such isolated services is the e-learning platform called TUWEL\footnote{\url{https://tuwel.tuwien.ac.at/}}, combining moodle\footnote{\url{https://moodle.org/}} and the central information system called TISS\footnote{\url{https://tiss.tuwien.ac.at/}}. These services provide course information and material, but are intended for different purposes. Whereas TISS focuses mainly on administrative functionality, TUWEL supports the interaction between teacher and student. 
Adding additional services which, for example, synchronize deadlines and registration dates is costly due to the fact that the information is separated over different isolated sources and not easily accessible. 

In this section we give a short overview of a prototypical architecture for the provision, processing, transformation and exposition of Linked Open Data. First, a high level view of the overall architecture is given. Then stakeholder specific concerns are discussed, highlighting relevant data sets and application types. In the remainder of this section, challenges for the design of a Linked Open Data solution are identified. 
\subsection{The big picture}
In this sub-section we provide an overview of a high level architecture for a university wide publication framework which includes
\begin{enumerate*}[label=(\roman*)]
  \item \textit{Requirements for Applications}, 
  \item \textit{Data Provision}, 
  \item \textit{Development of an Application} and an
  \item \textit{Application Use Case}
\end{enumerate*}
which is illustrated in Figure~\ref{fig:lod-architecture}. 
\begin{figure}[H]
	\centering \includegraphics*[width=.8\columnwidth]{lod_architecture.png}
	\caption{High level architecture of a system using Linked Open Data}
	\label{fig:lod-architecture}
\end{figure}
This mental model consists of several inbound interfaces indicated as arrows pointing from outside the system boundary and outbound interfaces indicated as arrows pointing from inside the system boundary.

Inbound interfaces are listed below:
\begin{itemize}
	\item \textit{User Requirements}~\\
	Understanding user requirements is an essential part of the software development process. However, specifying requirements is not an easy task due to constant changing needs of stakeholders. Not only adapting to stakeholders can become challenging, novel technologies and tools are critical factors to the success of a software product. Due to the open nature of Linked Open Data privacy concerns and legal issues should be already considered in the requirements as misunderstandings are hard to fix in later phases. 
	\item \textit{Existing Resources}~\\
	In the early days of information system there were certainly any concerns about interoperability as huge mainframes work in isolation. With the rise of the World Wide Web distributed, heterogeneous systems emerged, raising integration costs with legacy systems. At the Vienna University of Technology there are a few existing data sets (e.g. contact information of the university staff) which need to be integrated. 
	\item \textit{Stakeholders}~\\
	Stakeholders are groups, organizations or individuals who take part in the software development process. Stakeholders have an interest in the success of a software product. The most important examples are customers, (project) management and testers. 
	\item \textit{Application Developers}~\\
	Developers create new software products or modify existing ones by implementing (end-user) requirements. 
	\item \textit{LOD Experts}~\\
	Creating remarkable software products in specific domains requires people with knowledge superior to those of application developers. Cooperation between LOD experts, application developers and customers is a key factor in the success of a software product. 
\end{itemize}

Outbound interfaces are as follows:
\begin{itemize}
	\item \textit{Linked Open Data}~\\
	The fundamental concept of our proposed model is data connected by links. However, the model does not make any assumptions about the actual data representation format. The data is generated by a data provision process. There exist (semi-) automatic and manual transformation methods depending on the machine-readability of existing data. Data enrichment and interlinking procedures are applied to data sets to generate metadata properties~\cite{inproceedings:soa-architecture}. 
	\item \textit{Application}~\\
	A program or application is the outcome of a comprehensive development phase. The user interface is integrated into the application making it the only component where users directly interact with. In modern information system the application not a monolithic block installed on a client computer, but rather multiple instances running on different physical machines. 
	\item \textit{End-User Result}~\\
	After a successful launch users interact with the application in many ways. They click on buttons, navigate though menus and fill in forms. All these actions are done by the user interface which is mostly a visual representation of the application state. 
\end{itemize}

\%\%General Description of the Architecture\%\%
\subsection{Stakeholder specific concerns}
\subsection{Implementation challenges}

\section{Conclusion and Future Work}

\newpage
\bibliographystyle{plainnat}
\bibliography{references}

\newpage
\appendix
\section{Questionnaire}
\includepdf[pages=-]{questionaire_stefan.pdf}


\end{document}
