\documentclass{article}

\usepackage{url}
\usepackage{graphicx}
\usepackage{listings}
\usepackage{verbatim}
\usepackage{xspace}
\usepackage[left=3cm,right=3cm,top=2cm,bottom=1.5cm,includefoot]{geometry}

\usepackage{float}
\usepackage{tabularx}
\usepackage{hyperref}
\usepackage[english]{babel}
\usepackage[backend=bibtex, natbib=true]{biblatex}
\usepackage{amsmath}
\usepackage{csquotes}
\usepackage[final]{pdfpages}
\usepackage{listings}
\usepackage{paperandpencil} 


\addbibresource{references.bib}

\lstset{
  frame=single,
  basicstyle=\ttfamily\footnotesize,
  breaklines=true,
  columns=fullflexible,
  keepspaces=true,
  postbreak=\raisebox{0ex}[0ex][0ex]{\ensuremath{\hookrightarrow\space}}
}


\setlength{\textfloatsep}{5pt plus 1.0pt minus 2.0pt}
\setlength{\floatsep}{7pt plus 1.0pt minus 2.0pt}
\setlength{\intextsep }{7pt plus 1.0pt minus 2.0pt}

\setcounter{secnumdepth}{3}
\setcounter{tocdepth}{3}

\renewcommand{\UrlFont}{\scriptsize}


\begin{document}
\renewcommand{\bibname}{References}

\pagestyle{plain}

\title{\textsc{\textbf{L}inked~\textbf{O}pen~\textbf{D}ata}\\in\\Administration\\at the\\Vienna University of Technology}
\date{}
\author{Stefan Gamerith$^a$, \textit{Author}\\
Kevin Haller$^b$, \textit{Co-Author}\\
Lukas Baronyai$^c$, \textit{Co-Author}\\
\texttt{\{e0925081$^a$,kevin.haller$^b$,e1326526$^c$\}@student.tuwien.ac.at}}

\maketitle
\thispagestyle{empty}

\begin{abstract}
% Background
Although many universities have already published their public resources according to the Linked Data principles there 
have not been any efforts at the Vienna University of Technology.

% Methods & Results
This paper shows international success stories and best practices 
regarding Linked Open Data principles in an educational environment. To investigate major benefits and challenges concerning application 
development as well as the adoption of Linked Data principles and possible use cases for end-user applications, we interviewed two 
administrative workers at the Vienna University of Technology. Furthermore, to support practitioners and developers a prototypical 
implementation of a publication framework for educational resources as well as a general architecture of an application implementing the Linked Data principles
is given. 

% Conclusion(s)
Using Linked Open Data is a promising way of enhancing and harmonizing processes at universities, thus creating application scenarios with a large and supportive target audience is recommended. To limit integration costs, using existing, therefore maintained and open data sets is preferred over creating new ones. 


\end{abstract}

\newpage
\tableofcontents
\thispagestyle{empty}

\newpage
\pagenumbering{arabic}

\section{Introduction}
While the pressure on governments and public organizations to release \textit{Open Data~(OD)} has significantly grown with the spread of information systems there has been also a need for \textit{linking} these data from various sources to understand the information in a contextual sense.

As OD includes non-confidential data any restrictions in distribution are prohibited and information is funded only by public money~\cite{article:janssen2012benefits}. The application domain for OD providers is not restricted by its nature in any way including from traffic, weather and public transport to name a few. However, exclusively exposing public assets is not enough, in addition establishing a feedback loop facilitates an ongoing adaption to the stakeholders concerns. 

The World Wide Web has proven great success in spreading knowledge of various data sources all over the world. The building blocks of the Web are documents and links building a shared, global and connected information source. This can be seen as the key success factor in its nearly unconstrained growth~\cite{report:jacobs-i-2004--a}. 
\textit{Linked~Data~(LD)} has adopted these principles of publishing and connecting data realized as machine-readable, structured data connected to various data sets which in turn are linked to different data sets. 

\citet{artivle:bernerslee-t-2006-1} developed a five star deployment scheme classifying OD. The scheme ranges from a one star rating covering proprietary data formats~(e.g. Portable Document Format~(PDF)) to a five rating including machine-readable formats using open standards with links to other data sets.

Although LOD offers universities new opportunities for providing unprecedented insight into its core activities and ease application development, a major \textbf{problem} is that \textbf{LOD has not been widely adopted by universities yet}. Even tough there are a few examples~\cite{url:linked-universities-members} of publishing university related data sets as LOD there has been little knowledge of its usage for publishing university related information. 

To answer the research questions stated in the next sub-section we conducted interviews at the Vienna University of Technology. In particular \textbf{the context of this work is Administration}, though there exist papers investigating similar research questions but with different context~\cite{article:baronyai_publishing_2016, article:haller_publishing_2016}. 

The remainder of this section states the addressed research questions, describes the contributions of this work and gives an overview of the structure of this paper.

\subsection{Research Questions}
The fundamental research question we investigate is:
\begin{displayquote}
\textit{How can Linked~Open~Data help to improve processes in a university context and how can it be successfully applied?}
\end{displayquote}
More concrete, this paper concentrates on the following four research questions:
\paragraph{RQ1: What are best practices regarding the applicability of Linked Open Data in university settings?}
At the time of writing this paper there are no established best practices for the use of LOD due to its little adoption in university settings. For this very reason it is crucial to identify strengths and limitations from previous experiences of using LOD as the foundation of information systems~\cite{url:linked-universities-members}. 
\paragraph{RQ2: What are major benefits and barriers for each stakeholder and what are useful use cases?}
We identified three different stakeholders \textit{Students}, \textit{Researchers} and \textit{Administration staff}.
However, \textbf{focus of this work are administrative workers}.
Since the success of any new technology highly depends on their acceptance concerns of each target group needs examination. Furthermore, use cases are important to showcase profits and shortcomings. 
\paragraph{RQ3: What are major challenges for the implementation of a Linked Open Data solution?}
As the implementation of a LOD solution is a time consuming task the knowledge of probable challenges from a technical perspective as well as from a management perspective is key to a successful adoption. 
\paragraph{RQ4: How would a prototypical implementation of a publication framework based on Linked~Open~Data look like?}
Among the definition of building blocks for a publication framework of university related assets a high level picture of the overall architecture needs to be drawn to give implementers and LOD experts a common understanding of the system. Next, defining sample ontologies representing selected assets of the university domain draws concrete examples of how LOD data might look like. 

\subsection{Methodology}
Finding an answer for the research questions above has lead to the following three methodologies:
\paragraph{A coordinated set of semi-structured interviews}
To answer research questions~(RQ) two to four, we interviewed a selected set of \textit{Administration staff}. Semi-structured interviews were selected as the means for data collection because they are well suited for exploring the impressions and interests of the interviewees like in a discussion while still following a defined structure. 
\paragraph{Literature Review}
Undertaking a literature review to justify scientific contributions and making sound conclusions is an established practice in any scientific community. Since our scientific work targets to the Semantic Web community we made some pre-assumptions about a basic understanding of the technologies and concepts used in that respective area. More specifically, we assume a basic understanding of the concept of an ontology and some pre-knowledge about ontology description languages. 
\paragraph{Conceptual System Design}
The development of applications based on LOD requires a methodology facilitating a common understanding of the overall system infrastructure. Therefore we designed a conceptual model of a prototypical implementation of a publication framework based on LOD. 

\subsection{Contributions}
The work in this paper mainly contributes to different aspects which need to be considered when designing and implementing an application driven by LOD.
More precisely, our contributions can be categorized into the following four areas:
\paragraph{1. Identifying best practices for Linked Open Data in university settings.}
Information systems needs to cope with vast amount of data nowadays growing the need to efficiently handle Linked Data as well. We gave a brief overview of existing research work in that area, in particular, we compared the benefits and shortcomings in existing LOD solutions. 
\paragraph{2. Finding benefits/barriers including additional use cases for stakeholders.}
As with every software project the very first phase of the Software Development Lifecycle~(SDLC) is \textit{evaluating of requirements}. As LOD driven software has additional requirements to the accessibility of data and their organization we investigated if the overhead compared to an established technology~(e.g. a database based solution) is feasible. A set of selected \textit{administrative workers} are interviewed at the Vienna University of Technology. Additionally we proposed several use cases emphasizing their point of view. 
\paragraph{3. Discovering possible obstacles for implementers of a Linked Open Data solution.}
As the application domain for a Linked Data is limited to the university context our work includes LOD driven applications resulting from our conducted interviews.
\paragraph{4. Sketching a prototypical implementation of a publication framework driven by Linked Open Data.}
The proposed publication framework covers the whole process of data provision, requirement analysis and application development designed for but not limited to university related assets.  
\subsection{Structure of this Paper}
This paper is structured as follows:

Section \ref{sec:related_work} provides a summary
of existing efforts regarding Linked Data principles in educational contexts. Section \ref{sec:benefits_and_challenges}
discusses the used methodology and results 
concerning RQ2 and RQ3 from selected administrative workers.
While section \ref{sec:publication_framework} shows a prototypical implementation for a publication framework of educational assets,
section \ref{sec:application_architecture} proposes a technical architecture for the application of Linked Open
Data principles in educational environments.
Finally, conclusions and future investigations are made in sections \ref{sec:conclusions} and \ref{sec:future_work}.

Whereas section \ref{sec:application_architecture} is particularly intended for readers familiar with software architectures, the remaining sections do not necessarily require
a deeper understanding. However, a general understanding of Semantic Web technologies for the whole paper is recommended.  


\section{Related Work}
\label{sec:related_work}
This section gives a brief overview of existing work relating to Linked Open Data~(LOD) in educational contexts. In particular, we investigate experiences and established practices in that area. Fortunately there has been plenty of work done especially around \textit{Linked Universities}~(section \ref{sec:linked_universities}), an alliance of european universities. Next, selected participants of that project are briefly discussed who had gained great experiences at publishing educational resources. Although beneficial, but many contributions lead to divergent data sets and concepts. The project \textit{LinkedUp}~(section \ref{sec:linkedup}) tries to combine several of those, facilitating a uniform view of these concepts.

\subsection{Linked Universities}
\label{sec:linked_universities}
With the ongoing trend of providing unrestricted access to educational resources several universities participated in \textit{Linked Universities}\footnote{\url{http://linkeduniversities.org/}}, an alliance of european universities aiming at exposing their data as Linked Data. Although some universities had already published their educational resources, data formats often diverge, raising third party integration costs. For this very reason \textit{Linked Universities} facilitates data aggregation and integration among universities. Cross-university collaboration and sharing of common resources helps in establishing a common goal, which is the creation of a global Web of university data. 

More specifically, the goals and visions are:
\begin{itemize}
	\item Definition of common \textit{vocabularies} used for defining university related resources
	\item Sharing Linked Data related \textit{recipes} and \textit{tools}
	\item Collecting \textit{experiences} from various university projects
\end{itemize}

The most prominent members among their main contributors are:
\begin{itemize}
	\item \texttt{Mathieu d'Aquin} - The Open University, UK
	\item \texttt{Carsten Kessler} - University of Münster, Germany
	\item \texttt{Tomi Kauppinen} - Aalto University, Finland
	\item \texttt{Stefan Dietze} - The Open University, UK
	\item \texttt{Christopher Gutteridge} - University of Southampton
	\item \texttt{Hannes Ebner} - Royal Institute of Technology (KTH) / MetaSolutions AB
	\item \texttt{Charalampos Bratsas} - Aristotle University of Thessaloniki, Greece
	\item \texttt{Oguz Dikenelli} - Ege University, Turkey
	\item \texttt{Jakub Klímek} - Charles University in Prague
	\item \texttt{Jorge Pantoja} - Universitat Pompeu Fabra
\end{itemize}

\paragraph{Shared vocabularies and data sets}
The list below shows common vocabularies\footnote{\url{http://linkeduniversities.org/lu/index.php/vocabularies/index.html}} used for describing educational and university related resources together
with the involved concepts and recommendations for usage. 
\begin{itemize}
	\item \texttt{AIISO - an Academic Institution Internal Structure Ontology}~\\
	The Academic Institution Internal Structure Ontology~(AIISO)\footnote{\url{http://vocab.org/aiiso/}} offers a schema to describe the organizational structure of a university. Its intended usage is together with
	FOAF describing human relations, Participation Schema~\footnote{\url{http://vocab.org/participation/schema}} describing the roles that people play within groups, and AIISO-Roles\footnote{\url{http://vocab.org/aiiso-roles/schema}} describing common roles found in academic institutions and higher education. 
	\item \texttt{BIBO - The Bibliographic Ontology}~\\
	The original idea behind the adoption of the Bibliographic Ontology~(BIBO)\footnote{\url{http://bibliontology.com/}} was to express citations and bibliographic relations using a general concept like an ontology. Its usage is not limited to bibliography, but any kind of document describable by RDF. The ontology was originally created by Fredercick Giasson and Bruce D'Arcus, though its specification is in constant change and evolving. 
	\item \texttt{LSC - Linked Science Core Vocabulary}~\\
	The Linked Science Core Ontology~(LSC)\footnote{\url{http://linkedscience.org/lsc/ns-20111129/}} is targeted at relating objects in science to time, space and themes. In particular its design facilitates the definition of scientific assets including elements of research, their context and inter-relations. However, LSC is a collection of basic concepts in scientific publishing, hence the word \textit{core} in its name.  
	\item \texttt{TEACH - Teaching Core Vocabulary}~\\
	The Research Core Vocabulary\footnote{\url{http://linkedscience.org/teach/ns-20130425/}} is a simple ontology describing educational courses and staff.
	The main entry point is represented by the namespace prefix \textit{http://linkedscience.org/teach/ns}.
	Mixing with other ontologies 
	extends its applicability, though basic concepts of courses or seminars are covered. Some these related ontologies are \texttt{MLO} - an ontology for learning opportunities, 
	\texttt{FOAF} - a friend-of-a-friend ontology, the \texttt{Time Ontology}\footnote{\url{https://www.w3.org/TR/2006/WD-owl-time-20060927/}} and
	\texttt{AIISO} - an Academic Institution Internal Structure Ontology. 
\end{itemize}

Listing~\ref{lst:teach_example} shows a real world example\footnote{\url{http://data.aalto.fi/page/id/courses/noppa/course_Puu-0.3310}} of the ontologies listed above. 
It shows a course taught at the Aalto University\footnote{\url{http://www.aalto.fi/}} organized in autumn 2014. 
\begin{lstlisting}[caption={RDF/XML representation of an educational course for industrial symbiosis and industrial ecology concepts taught at the Aalto University using ontologies proposed at Linked Universities},label={lst:teach_example}]
  <aiiso:Course rdf:about="http://data.aalto.fi/id/courses/noppa/course_Puu-0.3310">
    <teach:arrangedAt rdf:resource="http://data.aalto.fi/id/courses/noppa/lecture_48700"/>
    <teach:academicTerm>I (NB! The course will be organized for the last time in autumn 2014.)</teach:academicTerm>
    <teach:hasAssignment rdf:resource="http://data.aalto.fi/id/courses/noppa/assignment_Puu-0.3310_0"/>
    <teach:arrangedAt rdf:resource="http://data.aalto.fi/id/courses/noppa/lecture_48461"/>
    <teach:arrangedAt rdf:resource="http://data.aalto.fi/id/courses/noppa/lecture_46879"/>
    <foaf:homepage rdf:resource="https://noppa.aalto.fi/noppa/kurssi/Puu-0.3310"/>
    <teach:arrangedAt rdf:resource="http://data.aalto.fi/id/courses/noppa/lecture_47564"/>
    <teach:hasAssignment rdf:resource="http://data.aalto.fi/id/courses/noppa/assignment_Puu-0.3310_1"/>
    <teach:arrangedAt rdf:resource="http://data.aalto.fi/id/courses/noppa/lecture_45862"/>
    <teach:arrangedAt rdf:resource="http://data.aalto.fi/id/courses/noppa/lecture_46880"/>
    <teach:arrangedAt rdf:resource="http://data.aalto.fi/id/courses/noppa/lecture_45861"/>
    <teach:grading></teach:grading>
    <dc:language>en</dc:language>
    <teach:arrangedAt rdf:resource="http://data.aalto.fi/id/courses/noppa/lecture_47020"/>
    <teach:courseTitle xml:lang="en">Industrial Symbiosis</teach:courseTitle>
    <teach:arrangedAt rdf:resource="http://data.aalto.fi/id/courses/noppa/lecture_46680"/>
    <teach:arrangedAt rdf:resource="http://data.aalto.fi/id/courses/noppa/lecture_46219"/>
    <aiiso:code>Puu-0.3310</aiiso:code>
    <teach:courseDescription xml:lang="en">Introduction to industrial symbiosis ...</teach:courseDescription>
    <teach:arrangedAt rdf:resource="http://data.aalto.fi/id/courses/noppa/lecture_88150"/>
    <teach:ects>5</teach:ects>
  </aiiso:Course>
\end{lstlisting}

\paragraph{Data conversion and transformation tools}
There has been put quite a lot of effort into collecting tools and software platforms commonly used to expose Linked Data at Linked~Universities. 
These software provide generic tools for data conversion and transformation. RDF has been chosen as the main target format, allowing further processing by general purpose XML manipulation software~(e.g. XPath evaluators and parsers).

The list below shows conversation and transformation tools inspired by the LUCERO Project\footnote{\url{http://lucero-project.info/}}, headed by 
Mathieu d'Aquin at the \textit{Open University}~\cite{inproceedings:open_university}.
\begin{itemize}
	\item \texttt{Triplify}~\\
		Still, a large amount of persistent data is stored in a relational database. Together with the SQL language these technologies dominated the way how data is structured and organized. A strict database schema makes data transformation difficult, however Triplify\footnote{\url{http://triplify.org/}} tackles these problems by converting relational data into \textit{subjects}, \textit{predicates} and \textit{objects}, hence its name. 
	\item \texttt{GRDDL}~\\
		Traditionally Extensible Stylesheet Language Transformations~(XSLT) are used for altering or converting XML documents. As XML and RDF share a common base~(e.g. RDF documents use XML based syntax) XSLT seems to be the first choice, however as RDF descriptions become complex transformation, is often infeasible. The \textit{Gleaning Resource Descriptions from Dialects of Languages}~(GRDDL)\cite{jour:grddl} serves exactly that purpose by making it possible to declare that XML documents include data compatible with RDF. 
	\item \texttt{Open Refine}~\\
		The data formats of choice for tabular organized data are CSV and Excel files. Transformation of such formats targeted at facilitating sharing between people can become quite challenging, especially for proprietary data formats. Open Refine\footnote{\url{http://openrefine.org/}} (formerly Google Refine) is a tool which eases data conversion and exploration from many different sources including Excel, CSV and Google Spreadsheet. Although it was not originally intended to support RDF export there has been created a tool\footnote{\url{http://refine.deri.ie/}} which adds RDF export as well as including some tools to connect tabular content to external linked datasets. 
\end{itemize}


\paragraph{Data sets and Endpoints}
All universities participating in Linked Universities provide endpoints for all different kinds of Linked Data.

The Linked Universities project provides a list of participating universities together with endpoints offering Linked Open Data. 
Unfortunately up to the time of writing this paper not all endpoints listed at the member page\footnote{\label{LOD_endpoints}\url{http://linkeduniversities.org/lu/index.php/datasets-and-endpoints/index.html}} are online. 

The table below shows selected universities together with online SPARQL endpoints and data catalogs: 

\begin{table}[H]
	\begin{tabularx}{\textwidth}{l|X|X}
		University & Data Set & SPARQL Endpoint \\
		\hline
		Open University & \url{http://data.open.ac.uk} & \url{http://data.open.ac.uk/query}\\
		University of Southampton & \url{http://data.southampton.ac.uk} & \url{http://sparql.data.southampton.ac.uk}\\
		Aalto University & \url{http://data.aalto.fi} & \url{http://data.aalto.fi/endpoint}\\
		University of Muenster & \url{http://data.uni-muenster.de} & \url{http://data.uni-muenster.de/php/sparql}\\
		University of Pompeu Fabra & \url{http://data.upf.edu/en/linked_data} & \url{http://data.upf.edu/en/sparql}\\
	\end{tabularx}
	\caption{SPARQL endpoints and data catalogs of selected members of Linked Universities}
	\label{table:sparql_data_catalog_linked_universities}
\end{table}

While the University of Pompeu Fabra provides access to just academic publications, others offer additional data sets covering academic 
courses, publications, research projects, places,  buildings, researchers,  staff, organizational structure, news and events. The most
complete data is provided by the Open University, a university located in the UK aimed at offering students a unique learning experience with
the support of information systems. In addition to the data sets described above the Open University provides audio/video recording of academic courses 
and social media related data sets as channels, playlists and posts from \url{http://audioboo.fm} and playlists as well as videos from \url{https://www.youtube.com}. 

\subsection{LinkedUp project - Linking Web Data for Education}
\label{sec:linkedup}
The LinkedUp project\footnote{\url{http://linkedup-project.eu/}} is a project funded by the European Commission aimed at pushing forward the exploitation and adoption of Linked Open Data in educational organizations and institutions. This project started on the 1st November 2012 and lasted about 2 years. 

More specifically, the projects main objectives are:
\begin{itemize}
	\item \texttt{Open Web Data Success Stories}~\\
	The identification of innovative success stories using tools and data sets in the education sector helps at spreading knowledge and awareness of Linked Data 
	principles. 
	\item \texttt{Web Data Curation}~\\
	The collection of relevant educational data sets facilitate the integration of third party data and applications. 
	\item \texttt{Evaluation Framework for Open Web Data Applications}~\\
	Evaluation of successful large scale applications driven by Linked Open Data promotes the adoption of Linked Open Data especially in educational contexts. 
	\item \texttt{Technology Transfer in the Education Sector}~\\
	The promotion of Linked Open Data technologies is an explicit goal of the LinkedUp project. 
\end{itemize}

The outcome of the LinkedUp project were several so-called \textit{"deliverables"}\footnote{\url{http://linkedup-project.eu/resources/deliverables/}}. 

Below, there is a list of the most relevant deliverables:
\paragraph{State of the art and data assessment}~\\
In this report~\cite{url:linkedup_state_of_the_art_and_data_assessment} a comprehensive study on the developments in the field of Linked Data and
related fields of educational data mining and learning analytics is given. The report resulted from a collaboration between the Leibniz University of Hannover, the Open Knowledge Foundation, the Open Universiteit Nederland, the Open University, Elsevier and Exact Learning Solutions. The authors first briefly describe fundamental technologies and concepts, namely Linked Data, educational data mining and learning analytics used throughout the paper. Several candidates representing each of the two basic data mining principles~(\textit{data harvesting} and \textit{distributed search}) for heterogeneous data sets are presented. The field of learning analytics focuses on techniques targeted at analyzing and understanding educational learning processes and environments. A complete discussion on this topic though, was out of scope since this is a relatively new research area. Next, challenges and barriers concerning the heterogeneity of open educational resources are investigated. Techniques trying to solve isolated data sets including \textit{schema mapping} and \textit{classification and clustering} are proposed. Finally, data sets from the LinkedUp repository\footnote{\url{http://data.linkededucation.org/linkedup/catalog/browse/}} and possible legal obstacles and solutions to these are discussed. 

\paragraph{The LinkedUp Challenge(s)}
The LinkedUp Challenge\footnote{\url{http://linkedup-challenge.org/}} was organized as three separate challenges \textit{Veni}, \textit{Vidi} and \textit{Vici} ending on October 2013, April 2014 and November 2014 respectively. While the goal of the first competition~(Veni) was developing a \textit{"prototype or demo that uses linked and/or open data for educational purposes"}, the Vidi challenge was targeted at finding \textit{"innovative and robust prototypes and demos for tools, which still may contain bugs, as long as it has a stable set of features and there is some proof that us can be deployed on a realistic scale"}. The rationale of the latter, yet more advanced challenge was to build \textit{"advanced prototypes and tools that should be mature and stable; it should be used or have been used by a fair amount of users on a realistic scale"}~\cite{url:linkedup_lnikedup_challenge_results}. 
The combined goal though was to promote creativity and innovation in ways to mash-up and link educational resources and services. In addition, companies, universities and government agencies were encouraged to share and (cross-)link to educational and non-educational assets. 

The winner for Veni was \texttt{Polimedia}\footnote{\url{http://www.polimedia.nl/}}, an application facilitating large-scale, cross-media analysis of the coverage of political events.
\texttt{TuvaLabs}\footnote{\url{https://tuvalabs.com/}} won the first place of the Vidi competition aimed at transforming open data into opportunities for meaningful teaching and learning, equipping teachers and students with high quality, consolidated data sets. Finally, \texttt{Flax}~(Flexible Language Acquisition)\footnote{\url{http://flax.nzdl.org/}},
the winner of the Vici competition, was designed to automate the production and delivery of interactive digital language collections and targeted to non-expert users~(e.g. language teachers, language learners, subject specialists, instructional design and e-learning support teams). 

\paragraph{Evaluation Framework}
\citet{url:linkedup_evaluation_framework} proposed an evaluation framework throughout the three open educational data competitions Veni, Vidi and Vici, described in detail in the paragraph above. The aim of the evaluation framework was to evaluate the usefulness, usability, acceptance and appropriateness of the contributions to the LinkedUp challenges. Therefore experts had to define and/or refine assessment criteria and indicators for measuring the quality of the framework. Practically the frameworks usefulness and ease of use was tested through a questionnaire and interviews. A detailed questionnaire for each of the three challenges as well as the final version of the evaluation framework are accessible at \cite{url:linkedup_evaluation_framework}. 

\subsection{Summary}
In this section a brief outline of the relevant research fields for this paper was presented.

In the beginning of this section the project \textit{Linked Universities}, an alliance of several universities, was introduced. The purpose of that collaboration was spreading the knowledge and adoption of Linked Data principles across universities borders. We quickly discussed shared vocabularies and data sets, introduced useful data conversion and transformation tools and finally listed selected SPARQL endpoints and data catalogs.

In the remainder of this section an interesting project called \textit{LinkedUp} and supported by the European Commission was briefly introduced. While being in some sense related to Linked Universities~(e.g. sharing similar goals), the LinkedUp project also covers practitioners and implementers viewpoints. That was specifically addressed by the LinkedUp challenge(s), three competitions intended at pushing the exploitation and adoption of Linked Open Data in educational organizations and institutions from a researchers as well as practitioners point of view. 
 
\section{Benefits and Challenges}
\label{sec:benefits_and_challenges}
In this chapter we figure out the driving factors of publishing university related assets as Linked Open Data and introduce sample applications built on top of it. First, we describe the used methodology~(section \ref{sec:methodology}) to show benefits and challenges using LOD at the Vienna University of Technology. Second, we discuss concrete topics of the interview, hence, potential needs, benefits and challenges are examined. We interviewed \textbf{2 representatives} of \textbf{administration} from the Vienna University of Technology. Next, we compare benefits against shortcomings by giving concrete use-cases to show under which circumstances LOD solutions are preferable to traditional ones (e.g. a database based implementation)~(section \ref{sec:results}). 

\subsection{Methodology}
\label{sec:methodology}
In this sub-section we describe the Vienna University of Technology and interviewed people more thoroughly, elaborate on the interview design, analysis and threats to validity.
\subsubsection{Interview context}
\label{sec:interview_context}
All interviews were taken at the Vienna University of Technology~\cite{url:university-of-technology-vienna}. The university was founded in 1815 as Imperial and Royal Polytechnical Institute. 160 years later~(1975) the university was renamed to its current name \textit{University of Technology}. At the time of writing the majority of the faculties are located at the fourth district in Vienna with nearly 9,000 rooms spreading over 276,000 square meters. 

The Vienna University of Technology provides a website of all organizational units falling into the category Administration~\&~Services~\cite{url:university-list-of-org-units}. For the evaluation of this work several departments were contacted by email. However, the majority of them declined participating because either they did not feel comfortable answering questions because of the technical concentration of the interview (e.g. Semantic Web in general and Linked Data in particular) or they were too busy in the restricted time frame~(4~months) the interviews were planned. However, we were forwarded to representatives of the organizational units \textit{Research and Transfer Support}~\cite{url:university-research-transfer} and \textit{Department for Studies and Examinations}~\cite{url:university-studies-and-examinations}. 
\subsubsection{Interview design}
As a solid preparation phase is the foundation of high quality results a considerable amount of time should be invested. Conceptually there exist three different types of interviews: \textit{Unstructured interviewing}, \textit{Semi-structured interviewing} and \textit{Fully-structured interviewing}~\cite{book:bernard-antropology-semi-structured-interview}. 
Due to the nature of unstructured interviews they are comparable to discussions, not following specific guidelines. In contrast, data collection in semi-structured and fully-structured interviews is typically documented by questionnaires. \citet{article:harris2010mixing} describe the difference between these two interview types as, while in fully-structured interviews the participant responds to predefined answers~(e.g. choosing from options) semi-structured interviews also focus more on testing interviewees responses motivating them to provide more background information and clarification. Due to the small number of interviews this paper primarily focus on understanding interviewees needs~\cite{book:miles2005handbook}.

The questionnaire is composed of three major parts:
\begin{enumerate}
	\item General questions targeted at getting a better understanding of the participants. It includes exploring prior knowledge about LOD in particular and information systems in general. 
	\item Analyzing stakeholders thoughts and opinions about presented example applications built using LOD motivating them to provide use cases for new applications. 
	\item Exploring existing and additional sources for data provision. 
\end{enumerate}


The list represents in which order the parts of the questionnaire were addressed during the interview, though there is no need for keeping this order in semi-structured interviewing.
A more detailed and complete image of the questionnaire can be taken from the appendix.

In the next few lines the interview guideline is described briefly.
First, the research team was introduced to the interviewee as well as the research objective. Afterwards, the moderator asked the interviewee about their background. Due to the level of expertise in the field of LOD the moderator was able to decide whether an introduction to LOD is necessary or not. Then the last two parts were covered discussing open questions concerning specific research goals.

The interviews were conducted in teams of two researchers. One researcher in the role of the moderator had the task to collect data by writing valuable contributions onto a logging sheet. This way, the interviewer can fully concentrate on listening and asking questions. An alternative approach is recording the whole conversation but in this case a single researcher has to spend more time (in worst case two times more) to extract important contributions. However, no other person would be required and recording can be used for later analysis.
Since we had enough human resources we decided for the former approach. 

\subsubsection{Data Analysis and Validity}
Semi-structured interviews mix open and closed questions allowing qualitative and quantitative analyzing methods for examining the interview outcome~\cite{article:runeson2009-interview-guidelines}. Quantitative data analysis is measurable and objective typically generating numerical values. Qualitative data analysis is descriptive and subjective focusing on deriving results and drawing conclusions from collected data~\cite{book:yin2013case}. Due to the exploratory nature of the interviews a detailed description of qualitative and quantitative analysis methods were omitted here.  

A common threat to the validity of collected data is a biased view on the interview answers. Two countermeasures were chosen to reduce this risk. First, the interviews were conducted by multiple researchers. Second, the interview was recorded for later analysis. 

\subsubsection{Interviewee Background}
At the beginning of the interviews participants answered some questions regarding their background including daily work tasks and responsibilities at the university. Furthermore, interviewees estimated their level of expertise in the areas Information \& Communication Technologies and LOD.

The concrete results of expertise are twofold. 
On the one hand one participant has excellent expertise with information systems and some experience with Linked Data. Expressed in numerical form of a scale from 1 representing little expertise to 5 representing expert knowledge the interviewee estimated his competence with information system systems 5 and Linked Data 3. As a side note, he commented to the latter estimation that he had already worked with LOD applications but never developed one on his own. He justifies his estimation with his background in computer science. 

The other participant has some experience with information systems but little with Linked Data. More formally, she estimated her competence with information system systems 3 and Linked Data 2, meaning that she classified herself an ordinary office worker with average skills in information systems and eventually heard but never used applications driven by Linked Data. 

\subsection{Results}
\label{sec:results}
After giving a detailed description of the used methodology in the previous sub-section this sub-section contains interview outcome.
Due to the reasons given in section \ref{sec:interview_context} and the background of a participant this section contains rather general, applicable not only to administration but research and student affairs as well. Concrete, one participant is responsible for administration and research. Even tough more participant were contacted, interviews were refused because of limited time.

First, potential use cases for building applications and publishing data as Linked Open Data are identified including benefits and barriers. Second, benefits and challenges are compared with one another resulting in guidelines for the use of LOD. 

\subsubsection{Building Map}
At the time of writing this paper there exist a website\footnote{\url{http://www.wegweiser.ac.at/}} for viewing lecture rooms, offices and other geographical related data. These website includes content of all Austrian universities hiding detailed information as elevators, stairways and other points of interest as buildings near-by, pubs and coffee~shops. 
This data is particularly relevant for external visitors who are new to the campus. Furthermore, maps are published as a PDF-file or image making it impossible to include any real~time data such as the utilization of computer rooms and defect elevators.

Figure~\ref{fig:um-map-app} shows a map application built for the University of Economy in Vienna. 
\begin{figure}[H]
	\centering \includegraphics*[width=.8\columnwidth]{map_wu_wien.png}
	\caption{A map application of the Vienna University of Economy}
	\label{fig:um-map-app}
\end{figure}
Although there is no indication if the map uses Linked Data the user interface shows everything of what is achievable with LOD. 
In figure~\ref{fig:um-map-app} a search for the rector of the Vienna University of Economy has been initiated which results in highlighting his office and selecting the appropriate floor on the left side. In addition, a guided route is initiated by clicking an adequate link on the turquoise overlay box. 
\paragraph{Benefits}
All interviewed stakeholders responded positive to the idea of a building application as described above. In particular they liked the ability of highlighting specific rooms and navigating through different floors. In general, there are advantages for disabled people and external visitors as outlined in the beginning of this sub-section. 

\paragraph{Challenges}
Digital data is the foundation of any information system requiring approaches for converting non-machine-readable data to machine-readable formats. Unfortunately the Department of Construction and Technology\footnote{\url{http://www.gut.tuwien.ac.at/}} at the Vienna University of Technology hosts exclusively building plans created by Computer Aided Drawing~(CAD) software prohibiting the addition of metadata for further processing. 

Special attention needs data ownership and copyright issues. Despite the fact that the Austrian government passed a law\footnote{Informationsweiterverwendungsgesetz (IWG)} regulating how and which governmental data could be published there are still legal factors which need to be clarified, notably if data ownership is exclusively hold by public institutions. 
\subsubsection{Tool for statistical evaluation of student related data}
Unistats\footnote{\url{https://unistats.direct.gov.uk/}} has published statistical course data of universities and colleges across United Kingdom. These data is exposed under the Open Government License\footnote{\url{http://www.nationalarchives.gov.uk/doc/open-government-licence/version/2/}} allowing copying, publishing, distributing and transmitting data without any restrictions. Besides offering data about undergraduate courses, post-graduate courses, students satisfaction scores, post-graduate salaries and other similar information a course assistant supports prospective students in choosing the appropriate study. 

Figure~\ref{fig:unistats-uk} shows search results originating from a sample search query which selects all part time, computer science courses in England ending in a Bachelor degree with a sandwich year. 
\begin{figure}[H]
	\centering \includegraphics*[width=.8\columnwidth]{unistats_uk.png}
	\caption{Course assistant application of Unistats UK}
	\label{fig:unistats-uk}
\end{figure}

\paragraph{Benefits}
All interviewees found the course assistant useful. The application is particularly attractive for future students who need help in selecting certain courses for participation. However, participants had concerns about major challenges as described below.  
\paragraph{Challenges}
The primary data source is evaluation data calculated from online feedback data. Unfortunately there is often not enough data available to draw solid conclusions. However, this is a rather general problem of feedback related data analysis. 

Despite the fact that the Vienna University of Technology do offer mechanisms for rating courses feedback related data is not accessible from outside of the universities backbone net limiting the options of third-party data analysis. 
\subsubsection{Researcher Portfolio}
The Vienna University of Technology provides researchers a platform\footnote{TISS Forschungsprofil \url{https://tiss.tuwien.ac.at/fpl/}} for presenting their research portfolio. The intention is give external stakeholders an insight into a researchers portfolio. However, searching for specific topics or cooperations from a non-technical perspective is not feasible. In fact, the searching algorithm selects candidates by exactly matching one of the given search parameters with some predefined tags. These tags are words provided by the researcher which are in most cases domain specific and only known by a limited research community. 

\citet{article:publication-database-linked-data} provide showcases of a successful application of a publication databased driven by LOD. Even tough a publication database built using Linked Data has advantages over ones powered by traditional technologies there are some challenges with need to be taken into account. 
\paragraph{Benefits}
Scientific research often involves cooperation with third parties. A Linked Data solution easily brings researchers and partners together as similar competencies are merged being understandable by both. 

Linked Data relies on open standards facilitating the creation of third party applications (e.g. statistical evaluation software). Participants responded that extensive data visualization eases the identification of cooperations between research and external stakeholders. 
\paragraph{Challenges}
Keeping data in sync is almost in any case challenging. In particular if there is no way of a (semi-)~automatic data transformation process. Even though a central database would remove the burden of duplicate data sets on different places the problem of manual data feeding still remains. 

Legal privacy regulations are different from country to country making the maintenance of a global publication database difficult. Privacy regulations often prohibit data distribution even in restricted areas. 
\subsubsection{Summary}
In this section we described the conducted interviews as the primary methodology as well as their results. We used a technique called semi-structured interviewing for data collection. Furthermore, we gave arguments for data validity and discussed different data analyzation methods. Since our aim is to better understand the stakeholders needs and challenges the majority of questions are of open nature qualifying for qualitative data analysis. 

In the remainder of this section we introduced use cases for Linked Data applications and identified benefits and challenges in their design and implementation. In addition to the shown use cases we listed suggested additional use cases. Among the individual benefits for each use case we found challenges common to all proposed applications. Privacy and legal issues are major challenges not only to Linked Data but information systems in general. This is particularly true for Open Data. Thus, thoroughly clarifying data ownership and possible legal obstacles are key to a successful LOD solution.  

\section{Proposed technical Architecture}
\label{sec:technical_architecture}
A university has to manage a significant amount of knowledge, adding new information on a daily basis.
Such an environment of data provision is complex and includes areas like academic data and educational resources which have to be conform to stakeholders requirements. Traditionally \textit{Service Oriented Architectures~(SOA)} have been used to meet these needs. However, as the application domain grows many small and similar services tend to emerge. That phenomena can not only be observed at the Vienna University of Technology, but also at the Open University\footnote{\url{http://www.open.ac.uk/}}~\cite{inproceedings:zablith_consuming_2011}.

A major problem of evolving similar, independent services are diverging data formats and different service owners. Thus, knowledge and administrative information that has been collected by multiple services can not be easily interlinked. An example for such isolated services is the e-learning platform called TUWEL\footnote{\url{https://tuwel.tuwien.ac.at/}}, a combination of moodle\footnote{\url{https://moodle.org/}} and the central information system called TISS\footnote{\url{https://tiss.tuwien.ac.at/}}. These services offer course information and material, but are intended for different purposes. Whereas TISS focuses mainly on administrative functionality TUWEL supports the interaction between teachers and students. 
Adding additional services which, for example, synchronize deadlines and registration dates is costly due to the fact that the information is spread over different isolated sources and not easily accessible. 

In this section we give a short overview of a prototypical publication framework for exposing university related assets as Linked Open Data. At first, a high level view of a publication framework is given~(section \ref{sec:publication_framework}). Afterwards, a prototypical architecture facilitating publishing university related assets is introduced. Finally, two ontologies are introduced covering human interactions and the university domain as a whole~(section \ref{sec:ontologies}). 
\subsection{A publication framework}
\label{sec:publication_framework}
In this sub-section we provide an overview of a publication framework for exposing university related assets as Linked Open Data, as indicated in figure~\ref{fig:lod-architecture}, including:
\begin{itemize}
  \item \textit{Requirements for Applications}~\\
  These requirements emerge from existing resources~(e.g. course material, contact data) and stakeholders input. After gathering requirements there has to be investigated whether a solution based on Linked Data is necessary.
  \item \textit{Data Provision}~\\
  Unfortunately information is often not available in a machine-readable format complicating LOD provision. Conceptionally the generation of Linked Data is either done manually or (semi-)~automatic. While the latter is preferable for large data sets manual transformation is the only option if information is present in non digital formats~(e.g. printed copies of building plans).  
  \item \textit{Development of an Application}~\\
  The generation of requirements originating from stakeholders input and existing resources is done by the development of applications driven by LOD. Since the domain of Linked Data in particular and Semantic Web in general is relatively new to most developers experts are needed to support the development process. 
  \item \textit{Application Use Case}~\\
  Applications run in a particular environment or domain forming the so-called application use case. 
\end{itemize}

\begin{figure}[H]
	\centering \includegraphics*[width=.8\columnwidth]{lod_architecture.png}
	\caption{A publication framework driven by Linked Open Data}
	\label{fig:lod-architecture}
\end{figure}
The publication framework in figure~\ref{fig:lod-architecture} consists of several inbound interfaces shown as arrows pointing from outside of the system and outbound interfaces shown as arrows pointing from inside the of system.

Inbound interfaces are listed below:
\begin{itemize}
	\item \textit{User Requirements}~\\
	Understanding the user requirements is an essential part of software development. However, specifying requirements is not an easy task due to evolving requirements. Not only adapting to stakeholders can become challenging novel technologies and tools are critical factors to the success of a software product. Privacy concerns and legal issues should already be considered in the requirements as misunderstandings are hard to fix in later phases. 
	\item \textit{Existing Resources}~\\
	In the early days of information systems there were certainly any interoperability concerns as huge mainframes work in isolation. With the rise of the World Wide Web distributed, heterogeneous systems emerged, raising integration costs with legacy systems. At the Vienna University of Technology there are a few existing data sets (e.g. contact information of the university staff) which need to be considered. 
	\item \textit{Stakeholders}~\\
	Stakeholders are groups, organizations or individuals who take part in the software development process. Stakeholders have an interest in the success of a software product. Prominent examples are customers, (project) manager and testers. 
	\item \textit{Application Developers}~\\
	Developers create new software products or modify existing ones by implementing (end-user) requirements. 
	\item \textit{LOD Experts}~\\
	Creating remarkable software products in specific domains require people with knowledge superior to those of application developers and collaboration between LOD experts, application developers and customers.
\end{itemize}

Outbound interfaces are as follows:
\begin{itemize}
	\item \textit{Linked Open Data}~\\
	A fundamental concept of our proposed model is data connected by links. However, the model does not make any assumptions about the actual data representation format. The data is generated by a data provision process. There exist (semi-) automatic and manual transformation methods depending on the usage of open data formats. Data enrichment and interlinking procedures are applied to data sets to generate metadata properties~\cite{inproceedings:soa-architecture}. 
	\item \textit{Application}~\\
	A program or application is the outcome of a comprehensive development process. The user interface is integrated into the application making it the only component where users directly interact with. In modern information systems the application is not a monolithic block installed on a client computer, but rather composed of multiple instances running on different physical machines. 
	\item \textit{End-User Result}~\\
	After a successful launch users interact with the application in many ways. They click on buttons, navigate though menus and fill in forms. All these actions are supported by the user interface.
\end{itemize}

\subsection{Proposal of a prototypical architecture}
\label{sec:application_architecture}
In order to overcome the environment of data silos and to support the evolvement of a university-wide data space for public resources, a technical architecture following the Linked Data principles is proposed and discussed in this section. 

In order to transform datasets into Linked Data and publish it in an open way, a system architecture is required that faces the issues of diverging data sources and data freshness.\textit{``An important architectural pattern used in system development is the multi-tier architecture. It logically separates the functionality of the system in a number of layers and specifies the communication between those layers.''}\cite{simperl_using_2013}.

The proposed architecture is mainly inspired by the toolkit \texttt{Tabloid}~(Toolkit ABout Linked Open Institutional Data) by the LUCERO Project~\cite{url:lucero-tabloid}, a collection of tools, examples and documentations. The general principle is a system, which extract RDF data from existing data sources (section~\ref{subsubsec:sources}), load them into a triple store (section~\ref{subsubsec:store}) and finally expose them to the web (section~\ref{subsubsec:provision}). 

For illustration see the LUCERO workflow in figure~\ref{Fi:lucero_architecture}: 
\begin{figure}[H]
\centering \includegraphics*[width=.8\columnwidth]{lucero-workflow.png}
\caption[Example: the LUCERO workflow]{Example: the LUCERO workflow~\cite{url:lucero-tabloid}}
\label{Fi:lucero_architecture}
\end{figure}

A more generic architecture can be seen in figure~\ref{Fi:tec_architecture}:

\begin{figure}[H]
\centering \includegraphics*[width=.8\columnwidth]{lod_technical_architecture.png}
\caption[Generic high level technical architecture of a LOD system]{Generic high level technical architecture for providing Linked Open Data (adapted from EUCLID {\cite{simperl_using_2013}})}
\label{Fi:tec_architecture}
\end{figure}

\subsubsection{Collect \& extract data from sources}
\label{subsubsec:sources}
The first step must be to collect and extract the data from sources and transform them into a proper LOD format~(e.g. RDF). By now there are basically for every data format tools for transforming them into RDF, LinkedUniversities made a collection of them~\footnote{\url{http://linkeduniversities.org/lu/index.php/tools/index.html}}.

Below are the main tools listed:
\begin{itemize}
	\item \textbf{From Relational Databases}
	\begin{itemize}
		\item \texttt{Triplify}~\footnote{\url{http://triplify.org/}} is a tool which use SQL queries to generate RDF data from a relational database.
		\item \texttt{D2RQ}~\footnote{\url{http://www4.wiwiss.fu-berlin.de/bizer/d2r-server/}} is a tool which also use SQL queries but with the use of a mapping that relates the structure of a database to RDF triples. It transformers SPARQL queries at run-time into SQL queries using this mapping.
	\end{itemize}
	\item \textbf{From XML and RSS}~\\
	In terms of syntax, XML, RSS and RDF sharing the same base, so there have to be fewer efforts to be done to transform them syntacticly, commonly using XSLT. W3C recommended the \texttt{GRDDL}~\footnote{\url{http://www.w3.org/TR/grddl/}} language for this purpose.
	\item \textbf{From Tables and Spreadsheets}
	\begin{itemize}
		\item \texttt{Google Refine}~\footnote{\url{http://code.google.com/p/google-refine/}} with the RDF Extension~\footnote{\url{http://lab.linkeddata.deri.ie/2010/grefine-rdf-extension/}} is an easy way to clean, transform and explore data in a tabular format including MS Excel, Google Spreadsheet and CSV.
		\item Other tools like \texttt{Any23}~\footnote{\url{http://lab.linkeddata.deri.ie/2010/grefine-rdf-extension/} (at the moment of the work unavailable)} or \texttt{QUIDICRC}~\footnote{\url{http://any23.org/}(at the moment of the work unavailable)} providing simple transformation from CSV to RDF
	\end{itemize}
\end{itemize}

After extracting the data as RDF from there source, they need to be cleaned and interlinked with themselves and other data.

\subsubsection{Store the data}
\label{subsubsec:store}
Considering the performance of such a system and to avoid too much communication with the original data source, the center of the system is a triple store, where the extracted data are stored and from there exposed to the web. A triple store (or RDF store) is a purpose-built database, \textit{``designed to store and retrieve identities that are constructed from triplex collections of strings (sequences of letters). These triplex collections represent a subject-predicate-object relationship that more or less corresponds to the definition put forth by the RDF standard.''}~\citet{url:triplestore}. There are a lot of implementations of such a triple store, like 
\texttt{Sesame}~\footnote{\url{http://rdf4j.org/}}, 
\texttt{Jena TDB}~\footnote{\url{http://openjena.org/TDB/}}, 
\texttt{4Store}~\footnote{\url{http://4store.org/}} or 
\texttt{SwiftOWLIM}~\footnote{\url{http://www.ontotext.com/owlim/}}. The LUCERO project settled for the \texttt{SwiftOWLIM}, because it is ``\textit{free, scalable and efficient, and includes limited reasoning capabilities, which might end up being useful in the future}''~\cite{url:lucero-tabloid}.

\subsubsection{Expose the data to the Web}
\label{subsubsec:provision}
The last step is obviously to expose the data from the triple store to the web. For this purpose a SPARQL endpoint is commonly used, but there are also other way, e.g. resolvable URIs, the linked data API~\footnote{\url{http://code.google.com/p/linked-data-api/}} or an Open RESTful API. All of them make the data available so others can query and access them to use it for their application or to serve themselves as resource for linking it with other data.

\subsection{Stakeholder specific concerns}
\label{sec:ontologies}
Dealing with stakeholder related concerns involves the definition of domain specific ontologies. In particular ontologies covering the field of organizational data at the Vienna University of Technology. First, this sub-section introduces two ontologies intended but not limited for supporting the generation and evaluation of statistical data of the universities organization. 

\citet{jour:gruber} defines an \textit{ontology} as:

``\textit{An ontology is an explicit specification of a conceptualization.}''

\textit{Conceptualizations} are objects, entities or concepts that may or may not exist in the universe. In addition to that, the \textit{vocabulary} defines the relationships between those objects. In other words, the vocabulary defines the conceptual model of what can be represented. 
\citet{jour:owl} describe ontologies from a more practical point of view defining the three conceptual components of an ontology - \textit{classes}, \textit{instances} and \textit{properties}. Regardless of what concrete implementation of an ontology is used graphical representations~(e.g. graph) are preferred over textual ones to give a high level overview of the ontology inherent concepts. 

\subsubsection{University ontology}
Figure~\ref{fig:owl-univ1} shows an excerpt of an ontology covering the whole university as an organization.
\begin{figure}[H]
	\centering \includegraphics*[width=.8\columnwidth]{owl-univ1.png}
	\caption{A graph showing an ontology representing the university as an organization}
	\label{fig:owl-univ1}
\end{figure}
Unfortunately the graph in figure~\ref{fig:owl-univ1} shows just a subset of the overall ontology as the \textit{Protege}\footnote{\url{http://protege.stanford.edu/}} plugin \textit{OWLViz}\footnote{\url{http://protegewiki.stanford.edu/wiki/OWLViz}} only supports classes and sub-class relationships, omitting object and datatype properties. Its entire definition is retrievable under \url{http://swat.cse.lehigh.edu/onto/univ-bench.owl}. 

The ontology was originally published by \citet{article:university-ontology} proposing a benchmark for measuring a variety of properties and several performance
metrics. However, this ontology covers the whole university domain extending its areas of application~(e.g. for statistical purposes). 

\paragraph{Relevant data (sets) at the Vienna University of Technology}
At the time of writing this paper the Vienna University of Technology publishes statistical data about active enrollments, beginners and active students. These data is currently publicly accessible\footnote{\url{https://tiss.tuwien.ac.at/statistik/public_lehre}} through TISS. Figure \ref{fig:tiss-statistic} shows all accessible data.
\begin{figure}[H]
	\centering \includegraphics*[width=.8\columnwidth]{tiss-statistical-data.png}
	\caption{A screenshot of the publicly accessible data set for statistical evaluation}
	\label{fig:tiss-statistic}
\end{figure}
Technically this data is stored in a relational database as part of the TISS ecosystem. Transformation of these data set would require adding additional metadata to be conform to the ontology presented in figure~\ref{fig:owl-univ1}. Furthermore, as that ontology is of rather high level, fitting for almost any educational organization, some adoptions are needed in order to keep the current amount of data. For example gender specific data or any other student related data other than \textit{graduate student} are not covered. 

\subsubsection{Friend-of-a-friend ontology}
Probably the most prominent ontology in the field of describing people and their relations is the \textit{Friend-of-a-Friend~(FOAF)}\footnote{\url{http://xmlns.com/foaf/spec/}} ontology. Despite its simple use case FOAF implements three different concepts: \textit{social networks}, \textit{representational networks} and \textit{information networks}. 
While social networks represent human collaboration and relationships, representational networks reveal communication relationships~\cite{book:encyclopedia-social-network}, e.g. people mentioned together in a communication channel enable a representation of network among them. Last, information networks connect independently published descriptions constituting an inter-connected graph.

Figure~\ref{fig:foaf-ontology} shows the generated ontology graph after its import from \url{http://xmlns.com/foaf/spec/20140114.html}.
The graph below was automatically generated by \textit{Protege}\footnote{\url{http://protege.stanford.edu/}} with the help of the plugin \textit{OWLViz}\footnote{\url{http://protegewiki.stanford.edu/wiki/OWLViz}} neglecting object and datatype properties.
\begin{figure}[H]
	\centering \includegraphics*[width=.8\columnwidth]{foaf-ontology.png}
	\caption{The FOAF core ontology describing people and their relationships}
	\label{fig:foaf-ontology}
\end{figure}
The FOAF ontology is simple, yet expressive enabling the adoption of many other ontologies in similar contexts. In fact, it is the most widely used ontology on the semantic web~\cite{article:social-networking}. Listing~\ref{lst:rdf_xml_example} shows a simple example drawn from the FOAF-specification and encoded using the Resource Description Framework~(RDF)\cite{article:rdf}.

\begin{lstlisting}[caption={RDF/XML representation of a person using the FOAF vocabulary},label={lst:rdf_xml_example}]
<foaf:Person>
	<foaf:name>Dan Brickley</foaf:name>
	<foaf:mbox_sha1sum>241021fb0e6289f92815fc210f9e9137262c252e</foaf:mbox_sha1sum>
	<foaf:homepage rdf:resource="http://rdfweb.org/people/danbri/"/>
	<foaf:img rdf:resource="http://rdfweb.org/people/danbri/mugshot/danbri-small.jpeg"/>
</foaf:Person .
\end{lstlisting}

The example in listing~\ref{lst:rdf_xml_example} shows a person whose name is \texttt{Dan Brickley}. He has a homepage~(\url{http://rdfweb.org/people/danbri/}) and uploaded an image~(\url{http://rdfweb.org/people/danbri/mugshot/danbri-small.jpeg}). A randomly generated identifier has been added~(\texttt{mbox\_sha1sum}) to distinguish it from records with identical data. 

\paragraph{Relevant data (sets) at the Vienna University of Technology}
Due to the broad applicability and its vast incarnations a single use case can not be spotted. Rather, more complex adoptions have been proposed to describe a comprehensive set of more domain specific properties. Concrete, the project \textit{Semantic Campus}~\cite{inproceedings:semantic-campus} aimed at defining an ontology for describing a universities personnel in more detail. 
\subsection{Summary}
In this section we initially described the idea of a prototypical publication framework for exposing university related information as Linked Open Data. The framework was described in a rather general manner, consisting of \textit{Requirements for Applications}, \textit{Data Provision}, \textit{Application Development} and an \textit{Application Use Case}. Second, an architecture for a general purpose application driven by Linked Data principles was introduced. Third, two selected ontologies representing candidates for modeling a universities organization at high level and human relations respectively were presented.
Last, relevant data sets at the Vienna University of Technology were introduced, though the FOAF-ontology is mostly extended by more sophisticated ontologies rather than used exclusively. 
\section{Conclusions and Future Work}
The high-level goal of this work is to enhance and improve processes at the Vienna University of Technology. We have broken down this goal
into four distinct research questions:
\begin{itemize}
	\item \textit{RQ1: What are best practices regarding the applicability of Linked Open Data in university settings?}
	\item \textit{RQ2: What are major benefits and barriers for each stakeholder and what are useful use cases?}
	\item \textit{RQ3: What are major challenges for the implementation of a Linked Open Data solution?}
	\item \textit{RQ4: How would a prototypical implementation of a publication framework based on Linked Open Data look like?}
\end{itemize}
In the first part of this paper~(section \ref{sec:related_work}) we investigated the first research question and provided international best practices and success stories 
in educational environments. In the second part~(section \ref{sec:benefits_and_challenges}) we focused on benefits and challenges regarding the applicability as well as
the implementation of Linked Data principles for administrative workers at the Vienna University of Technology by conducting interviews, addressing the second and the third research question. In the final part~(section \ref{sec:technical_architecture}) we addressed research question four by giving a high level overview of a publication framework for educational resources at first~(section \ref{sec:publication_framework}) and then providing a general architecture for applications following Linked Data principles~(section \ref{sec:application_architecture}). 

In this section we discuss conclusions and contributions related to the four research questions we investigated (section \ref{sec:conclusions}) and present ideas for future work (section \ref{sec:future_work}). 
\subsection{Conclusions and Contributions}
\label{sec:conclusions}
In this section we focus on major conclusions and describe contributions we made. This sub-section is organized around our four major research questions.

As this work is highly influenced by the conducted interviews the global conclusion is that the most benefits will be achieved if Linked Open Data principles are implemented right from the beginning, meaning that every process is implemented with these principles in mind. Unfortunately this is almost never the case, therefore requirements and expectations for (end-user)~applications and supported processes have to be cut down. 

\subsubsection{Best practices regarding the applicability of Linked Open Data in educational contexts}
In the first part of this paper~(section \ref{sec:related_work}) we presented two international projects targeted at spreading knowledge and success stories regarding Linked Open Data principles in educational environments. We hereby conclude:


\textbf{Using Linked Open Data support is a promising way of enhancing and harmonizing processes at universities.}
In particular, these achievements have let to several contributions. One major contribution was the creation of ontologies~(e.g. AIISO) and tools~(e.g. Triplify) applicable in general university settings. 
Hence, a general guideline should be \textit{using already existing ontologies and tools instead of creating new ones} which cover similar concepts and domains.\\

\textit{Our contribution is to identify and describe these (international) best practices.}

\subsubsection{Major benefits, challenges and use uses for stakeholders}
In the second part~(section \ref{sec:benefits_and_challenges}) we guided our attention on benefits and challenges regarding the applicability as well as
potential use cases for administrative workers at the Vienna University of Technology. 
Our work resulted in the following conclusion:

\textbf{Use application scenarios with large and supportive stakeholder groups.}
It is much easier and profitable to build applications in a preferably large and supportive environment. Nothing hinders progress more than a skeptical target audience. 
As we have seen from interviews the group of administrative workers and researchers are small and manageable. Rather, to take advantage and spreading knowledge of Linked Open Data principles we suggest choosing students as target stakeholders. In addition, offering courses, provide students the ability to develop applications driven by LOD and helps in enhancing and maintaining LOD data sets. \\

\textit{Our contribution is to get a better understanding of the stakeholders needs at first and then identify potential use cases and aspects in realizing these.}

\subsubsection{Major challenges for implementing applications following Linked Open Data principles}
Also in the second part~(section \ref{sec:benefits_and_challenges}) we additionally provided challenges and obstacles in application development. 
Therefore, the following conclusion were drawn: 

\textbf{We recommend using existing, therefore maintained and open data sets as a foundation of new applications.}
The cost of integrating large data sets can become quite time consuming if independent domains are mixed. Even for smaller, more limited domains using existing data sets outweighs the benefits of creating new, fully customized data sets. If additional requirements arise modifying and/or adding to existing data is preferable. A cost minimizing strategy for keeping the data clean and manageable is to clean existing data sets before adopting them to Linked Data principles. \citet{artcle:rahm2000data} gives a more in-depth introduction into data cleaning approaches and challenges. \\

\textit{Our contribution is to identify potential challenges and obstacles in implementing applications driven by LOD.}

\subsubsection{Prototypical implementation of a publication framework}
In the final part~(section \ref{sec:technical_architecture}) we addressed research question four by introducing a publication framework as well as a general application architecture using Linked Data principles for university related resources. 
We come to the following conclusions:

\textbf{Applications making use of LOD should be built following a general architecture.}
There have already been efforts at building LOD applications at enterprise level, sharing a similar foundation. An agreed picture of a successful architecture facilitates establishing common vocabularies and understandings under developers, customers and management. \\

\textit{Our contribution is to provide a general architecture for LOD applications, hence supporting developers in creating these kinds of applications.}

\subsection{Future Work}
\label{sec:future_work}
Contributions and methods of this work is intended as an initial step to investigate the adoption of Linked Data principles and spread awareness and knowledge at the Vienna University of Technology. 

The following list summarizes planned future work:
\begin{itemize}
	\item \texttt{(Educational) courses aimed at spreading Linked Data principles}~\\
	Educational courses are great opportunities to facilitate building of LOD applications and maintaining related data sets. 
	\item \texttt{Building (end-user) LOD applications}~\\
	Application development is not limited to resulting from educational courses. Rather, due to the heterogeneity of universities data sets, a collaboration between external parties and university representatives is required.  
	\item \texttt{Consolidated report to executive board}~\\
	There will be a followup report, summarizing our contributions for students, researchers and administrative workers. This report is addressed to the executive board of the Vienna University of Technology. A side-product of this report will be a brief presentation about the overall result and achievements, planned in the near future. 
\end{itemize}
\newpage

%-----------------BIBLIOGRAPHY-----------------
\addcontentsline{toc}{section}{References}
%\bibliographystyle{plainnat}
%\bibliography{references}

\printbibliography[title={References to refereed scientific work}, keyword=refsw]

\printbibliography[title={References to non-refereed work}, keyword=nonrefw]

\printbibliography[title={References to websites}, type=misc]
%----------------------------------------------

\newpage
\appendix

\section{Questionnaire}
\documentclass[12pt,fleqn]{scrreprt}

\usepackage[top=2.5cm,bottom=2.5cm,left=2.5cm,right=2.5cm]{geometry}
\usepackage[utf8]{inputenc}
\usepackage{paperandpencil}
\usepackage{setspace}

\begin{document}

\fbox{\fbox{\parbox{.45\linewidth}
{\centering
\textsc{Questionaire}\\
\textsc{\textbf{L}inked \textbf{O}pen \textbf{D}ata}
}}}
\hfill
\parbox{.45\linewidth}{
\begin{doublespace}
\normalsize Name:\hrulefill\\
Department:\hrulefill\\
Date: \hrulefill
\end{doublespace}
}


\section*{General Questions}
\vspace{0.5cm}

\question{\bfseries Which organizational area would you associate your work with?}
%\horizontalthree{\upthree{Research}{Student Affairs}{Administration}}
\begin{answersA}
\item \ebigbox{Research}
\item \ebigbox{Student Affairs}
\item \ebigbox{Administration}
\item Other\linetext{}
\end{answersA}

\question{\bfseries How would you describe your area of responsibility? How would you characterize your \textit{daily} work tasks?}
\openthree

\question{\bfseries How would you say classify your level of experience with Information Systems?}
\horizontalfive{\upfive{Novice\footnotemark[1]}{Advanced Beginner\footnotemark[2]}{Competent\footnotemark[3]}{Proficient\footnotemark[4]}{Expert\footnotemark[5]}}{\downfive{}{}{}{}{}}

\question{\bfseries How would you classify your level of expertise with Linked Open Data?}
\horizontalfive{\upfive{never heard}{heard but never used}{used in small example}{used in practice}{Expert in LOD}}
{\downfive{}{}{}{}{}}
\footnotetext[1]{No experience with the situations in which they are expected to perform tasks.}
\footnotetext[2]{Has coped with enough real situations to note the recurrent meaningful situational components. }
\footnotetext[3]{Has already gained a considerable amount of routine to do tasks by deliberate planning to reach long-term goals. (e.g. do things in a organized way) }
\footnotetext[4]{Perceives situations as wholes, rather than in terms of aspects, and performance is guided by maxims.}
\footnotetext[5]{Has a vision of what is possible and uses analytical approaches in new situations or in case of problems.}
%% Taken from the "Dreyfus model of skill acquisition" %%
\newpage

\section*{LOD in Administration}
\vspace{0.5cm}

\question{\bfseries How useful do you find the improvements with Linked Data in building maps?}
\horizontalfive{\upfive{not useful}{}{useful}{}{extremely useful}}
{\downfive{1}{2}{3}{4}{5}}
Why?
\opentwo

\question{\bfseries How useful do you find publishing Linked Data for statistical evaluation?}
\horizontalfive{\upfive{not useful}{}{useful}{}{extremely useful}}
{\downfive{1}{2}{3}{4}{5}}
Why?
\opentwo

\newpage

\question{\textbf{Can you imagine similar projects at the Vienna University of Technology?}}
Yes, in particular:
\opentwo
No, because:
\opentwo
What benefits do you see?\\
\opentwo
What disadvantages or barriers do you see?\\
\opentwo
Can you imagine/recommend other roles/persons?\\
\opentwo

\newpage

\section*{LOD Applications}
\question{\textbf{Do you know any useful application of LOD which will help you in your daily work or will make internal procedures at the university more efficient?}}
\openfour
What benefits do you see?\\
\opentwo
What disadvantages or barriers do you see?\\
\opentwo
Can you imagine/recommend other roles/persons?\\
\opentwo
\newpage

\section*{LOD Data}
\question{\textbf{Do you know any other data regarding administration or in general which can be published as LOD?}}
\openfour
What benefits do you see?\\
\opentwo
What disadvantages or barriers do you see?\\
\opentwo
Can you imagine/recommend other roles/persons?\\
\opentwo
\newpage

\section*{LOD in general}
\question{\textbf{Do you have any other kinds of suggestions regarding to LOD in general or with focus on administration?}}
\openfour
\end{document}



\end{document}
